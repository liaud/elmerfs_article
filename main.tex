\documentclass[sigplan, 10pt]{acmart}
\usepackage[utf8]{inputenc}
\renewcommand\footnotetextcopyrightpermission[1]{}
\pagestyle{plain}

\settopmatter{printfolios=true}
\begin{document}

\author{Romain Vaillant}
\email{romain.vaillant@scality.com}

\title{ElmerFS: Building a file system using CRDTs}

\begin{abstract}
\end{abstract}

\maketitle


\section{Introduction}

\begin{enumerate}
    \item\textbf{TODO: Talk about POSIX earlier}
    \item\textbf{TODO: Insist on the GEO distributed part}
\end{enumerate}

File systems exist to abstract and give structure to the general problem of
storing information on a storage medium. Storage lifetime requirements can range
from storing for a few microseconds to an endless amount of time.

They also have a very wide range of applications. With our personal computers,
file systems only have to manage a couple of storage media and handle
concurrency locally between applications run by one single user. On a broader scale, file hosting services not only need to cope
with the usual problems that can be encountered with persistent media but also
deal with concurrency between multiple users acting on the same files and
directories across multiple data centers, with high
latencies between them due to their geographic distance.

Being a critical component, high availability for those services is also
a strict requirement. Even small failures can lead to hours being wasted waiting
for the service to go back to normal and sometimes without the possibility to
access a partially functioning interface, without the possibility to work
locally in the meantime.

On the opposite of file systems, there are object stores that persist,
often immutable, arbitrary blob of data. Those blobs are uniquely identified
under a common namespace and act as the sole unit of abstraction.
They have their own advantages, notably because they don't suffer from the burden
of a strict hierarchy and have a much limited set of available operation:
the simplest object store can only provide the put, get and delete operation operations whereas,
a file system, need  to provide all the operations to manipulate the tree hierarchy.
As a consequence, they are simpler in their structure and are easier to scale,
now making them a strong component in Cloud services.

However, hierarchical file systems are still one of the most approachable
user interface to organize and access data that exists today.
The challenges that they bring to be distributed, resilient, scalable
and efficient still exist.

Distributed systems are notoriously hard to build and harder to
build right. Here, we explore the path of building a plausible distributed file
system that is geos-distributed, highly resilient and scalable by using
\textbf{Conflict Free Replicated Data Types} (CRDTs) as our main framework:
data structures that don't need any coordination to be replicated. To ease our
development, we leverage \textbf{AntidoteDB}, a distributed database based
on CRDTs and we use the rust programming language,
a compiled language free of any runtime which enforces memory safety and
thread safety with its type system.

Our findings are that is is possible to build a file system spanning multiple
data-centers using CRDTs that can work under extreme failure and network conditions while
resolving complex concurrent user interactions with an intuitive user interface.

\section{Designing a file system}

\subsection{What a basic file system requires}

There are many ways to implement a file system depending on its usage. We
decided to design a distributed file system that follows  the POSIX standard.
More precisely, we implement a user-space file system using the FUSE protocol,
which avoids the need of creating a kernel module and all the intricacies
that comes along with it.

A basic distributed file system must provide both metadata operations and
content operations. Metadata operations are here to manipulate the file system
structure, permissions and retrieving some information. The content operations
are there to allow the user to persist an arbitrary amount of data to a given
file and then be able to read it.
It is also common to expect from distributed file system some amount
of resiliency and concurrency handling; at least for manipulating the
file system structure. For content operations,
a user should be able to write files that range from
a few bytes to multiple gigabytes.

Drawing from the POSIX standard, we will use in this document standard known terms that
we recall here:

\begin{itemize}
    \item \textbf{An inode}: It is the internal representation of a filesystem object:
    files, directories or symlinks, they all share a same common information
    set stored as an inode.
    \item \textbf{A link}: A reference to some inode. An inode that doesn't have
    any link, doesn't need to be kept, it can be deleted. Usually, all references
    are visible by the user.
    \item \textbf{An inode number}: Usually called ino, it is
    a 64bit unique id used to reference an inode.
    \item \textbf{A directory entry}: The named and visible representation of
    a link. In most cases, users reference inodes by using their names which
    is then resolved into an ino.
    \item \textbf{A block}: This is the smallest unit of storage available.
    Files always have a effective size that is aligned to the size of a block.
\end{itemize}

One of a key component of a POSIX file system is its tree hierarchy.
This tree has many complex properties, the ones that are the most common are:

\begin{enumerate}
    \item Nodes in this tree are links to inodes. A directory is a container
    of those links.
    \item Directories can have at most one existing link.
    \item All nodes are always reachable from the root directory.
    \item Cycles are possible only through symbolic links.
\end{enumerate}

\subsection{Assumptions and precise goals}

The goals of our implementation is to provide a plausible file system service
that is geos-distributed and always available. This means that we aim to implement
all basic operations that one might expect from a basic file system. Being
geo-distributed, we want our file system to be always available and provide good
latencies whatever the network conditions are. The file system must be able to work
in an active-active situation: two clusters are able to issues read, writes and metadata
operations at the same time without waiting for each other.

We aimed for a system that works first and that helps us to see what it takes
to use CRDTs for real systems instead of focusing on throughput and raw performances.
Of course, waiting for an unbearable amount of time is still considered as a failure.
In summary, most of the design focus was done on the metadata operations
and their semantics rather than their optimizations.

One can expect the following properties from our file system:

\begin{itemize}
    \item \textbf{No locking}: One way to handle concurrency is to use distributed locking
    to serialize operation applied on the relevant objects. CRDTs provide a way for us to escape
    this and to allow true concurrency without the need for any consensus protocol.
    We aim to make the locking optional and only applying it at the will of the user.
    All metadata and updates operations do not take any distributed
    locks by default.
    \item \textbf{Atomic operation}: We want to keep the behavior of POSIX file systems
    in which every operation is atomic.
    \item \textbf{Active-Active}: As stated above, we want updates to be able to
    come from any source at anytime, no matter how old the updates are.
    \item \textbf{Least surprise}: We don't want to perform too many actions
      or changes besides what the ones the users does. Meaning we want to
      avoid rollbacks or drastic structural changes behind the scene.
\end{itemize}

\subsection{A high level interface for CRDTs}

Like an application that can rely on an external database service to persist its
data preserving ACID constraints, here we use AntidoteDB to provide us with
tooling and infrastructure to handle CRDTs and their replication.
AntidoteDB by default uses the Cure protocol, allowing us to perform interactive
transactions that concern multiple objects with consistent read and writes.

Each operation in ElmerFS is wrapped inside an AntidoteDB
transaction. Any failure will abort the transaction preventing partial updates.
No other specific work has been done is this regard since all is handled
by the protocol just like when using transactions in common databases.

Using this infrastructure and abstraction, ElmerFS is only a thin layer between
the application interface (here POSIX) and the AntidoteDB database. We only
have to be concerned by the application main logic: chose the right
CRDTs for our structural representation and issue the correct
sequence of operation inside a transaction.

\subsection{Predictability in latencies}

As ElmerFS is a thin layer which rely on multiple roundtrip with AntidoteDB,
we always deploy ElmerFS and AntidoteDB side by side on the same machine.
AntidoteDB receives our requests, applies the CRDT transformations
locally and gives us a response as soon as it is complete.

This allows to reduce the latency between the user's operation and our response.
The local AntidoteDB is expected to join permanent clusters with multiple
running AntidoteDB instances. The latency between the local AntidoteDB and
the cluster might be high, but the perceived latency for the user stays
relatively low. Since we do not require mandatory locking, we never have to
wait for a majority of the quorum, everything is as local as possible.

\section{Building blocks for building a file system}

\subsection{Composing CRDTs}

AntidoteDB provides us with a few basic CRDTs that we can use and compose together
to represent the state of the application. From a high level point of view,
AntidoteDB is a key-value store where each value is a CRDT.
When dealing with a database, we have to map
the in-memory application state to CRDTs.
In a statically typed programming language such as Rust, it means
that we must map our structures and enumerations to some CRDTs.

For structures, we were naturally leaned towards the type map CRDTs.
In AntidoteDB, there are multiple kind of maps to chose from each with specific
convergence properties. It is up to the application to chose what would be the best fit
for its business logic.

Grow Only Maps (GOMaps) are maps where the removal of elements is not possible.
At first, we thought that this would be a good fit to represent
our structures. Rust is statically typed, we cannot really remove one specific
field: either all fields are initialized or none of them. However, AntidoteDB
doesn't have the notion of existence and we cannot ask it to simply delete
a GOMap when the entity has been deleted (\textbf{TODO: Maybe explain why }).
Remove Win Maps (RWMap) are used instead, with the nice property that
we can only issue updates of specific fields and not worrying of
having structure partially existing as that would be the case
with Add Win Maps (AWMaps).

Most of the fields can be encoded as simple registers. The CURE (\textbf{TODO: ref!})
protocol, which is used by AntidoteDB, allows us to
perform read-write transactions spanning multiple CRDTs and ensure
that all registers stay consistent in respect to each other.

We will see in what follows that by only composing CRDTs we can represent
most of the file system state with complex convergence properties. In a way,
composing CRDTs is the tool for the application to create one final CRDT, which
is the application state. \textbf{Having the right abstracted and formalized CRDTs is
key to allow an application to model its state without having to
formalize one specialized CRDT for each domain and without having to bend
its domain logic in a diminishing way}.

\subsection{A common base}

We won't describe in details how ElmerFS identify its entities,
it simply computes a composite key from a kind
(inode, block, directory, ...) and an ino.

The ino generation is also very simple. Each instance reserves a predetermined
amount of ino beforehand and keeps them in memory. To ensure that multiple
instances do not produce the same ino, we have to use a lock. A counter CRDT
doesn't work for this (we have to read the value, not just incrementing it). In fact
this is the only place where a mandatory lock is used.
A sequence CRDT (\textbf{TODO: which one}) might be a solution but aren't available in
AntidoteDB.

A consequence of this approach is that inos are never recycled. We can never
be certain when a CRDT will never be used again as we allow divergence for an
not constrained amount of time. However, for this particular case, 64bit is a
large enough domain for this problem to be problematic for most common use
cases.

Our inode contains all metadata information that are common in POSIX:
permissions, owner, change, modification and access
time. As described above, we represent them using a RWMap.
We made the decision not to store data information of the file and the
symbolic link path alongside the metadata.

As only inodes represent the existence of entities, this separation leads to
potential leaks if some inode is deleted in one site and written to in another.
However, it is more a limitation of the API of AntidoteDB than CRDTs which
doesn't support partial read or update.

\subsection{The heart of a file system interface}

To represent the tree hierarchy, it is easy to see that just applying CRDTs merging semantics
without care can easily break the tree constraints.
There is active research (ref?) to provide a Tree CRDT with add, remove and move operations.
In ElmerFS we didn't implement one of such CRDT that would have ease our
development. We explored the path of only using RWMaps.

Even though we still have found any simple solution to prevent cycle apart from
keeping track of all rename operations and cancelling the
one in conflict in a deterministic way, we still manage to handle
most of the concurrent tree operations gracefully without distributed locks.
This solution isn't perfect, our choice to prevent cycles would be costly at
the application level.

\section{Embracing the CRDT merging semantics}

To handle most conflicting operations with only common
CRDTs, ElmerFS introduces the concept of an owner of each operation and
each owner has its own view of the current file system.

\subsection{Some problematic cases}

First, let's consider some problematic cases of the rename operation.
On a POSIX file system, a rename operation is expected to be exclusive
with respect to the renamed inodes.

% confusing
In a concurrent scenario, the first problematic situation
arises when two users try to rename distinct inodes to the same target location with the same name.
In that case, CRDT maps merging semantics will only keep one operation depending
in which order they arrived. In all cases one link will be lost and the inode
might become unreachable if the was the only link to that was referencing it.

Moreover, those name conflicts can also happen with all operations that
can create links: link, mkdir, mknod\dots

The second situation happens when a unique reference is renamed
by two users at the same time to two different locations. The standard behavior
is that one of the operation will fail. This poses two new problems: In a completely
asynchronous situation, how do we converge to a state that is meaningful to the
users and how do we preserve the reference count with an operation that
semantically doesn't affect it ?

\subsection{Keeping conflicts where they make sense}

In ElmerFS, we solve these problems by using an unique value that we call the
ViewID. This unique ID is used to identify the origin of the operations and to
use the merge semantic of map and set CRDTs to our advantage. Apart from
being unique, there is no particular requirement for this ID. In ElmerFS we used
the file system user id (uid) assuming that a user might not be
issuing operations from two different processes.
Another solution would have been to tie a unique id to each process
but we find that the mapping between the user and its actions is more
straightforward in the former solution.

In our application we use a RWSet CRDT to represent directories.
We depart from standard file system by not only storing a pair of ino
and name to represent a link but also by storing the ViewID alongside this
original pair.

If we take the previous concurrent case where two users tries to rename two
inode to the same location, with the ViewID associated with each name, we
ensure that we either don't have a conflict (no tuple in the set
share a same name entry) or that we keep all succeeded operations
in our application state: there might be two identical name in the set, but they
will have a different ViewID.

By adding uniqueness to our directory entries, we allow precise conflicts to bubble up to the application logic when they make sense.
When two user try to create a new file inside a same directory, we don't
really want one operation to fail afterward. As soon as the user created the
file and started writing it, it is crucial for the file system to keep this
file. Conflicts here allow us to choose how to present and how to manage
conflict without designing a specific CRDTs.

\subsection{Linking and reference counting}

By using the same mechanism we can also keep track of the number of reference
of each inode where a counter CRDT would obviously fall short. One can imagine two users
renaming the same file to a different location. In ElmerFS, we accept this kind
of operation but it means that this operation is similar to creating an additional link. We use a set CRDT to store all links of an inode. Inside a transaction,
we issue a remove operation of the previous link and an add operation of the new link. The remove operation will contains the exact tuple matched
(with the ViewID that created the link), When the CRDT converges, it receives
two identical remove operations, and two distinct add operations, leading to the right reference count.

Directories have the additional requirements that they can only have one unique
parent in the file system. To preserve this invariant, we also use a register
with last writer win semantics, to decide which reference is valid. Note that
we could also allow multiple link of a directories,
this is only a limitation of the POSIX interface.

\subsection{Interfacing with the user}

Interfacing with the user in case of non standard FS behaviors requires care.
While we solved some issues with the renames and links operation, we still
need to show to the user that the conflict exists and that it needs to be
resolved. Since in a FS the user is expected to reference entities by name,
we kept this idiom to resolve conflict.

In ElmerFS, a name is either a shorthand or a fully qualified (FQN).
What we call a FQN is a standard POSIX name (the prefix) and
the ViewID delimited by a special character.
For example, in ElmerFS we use the uid for convenience and
we allow user to specify the username instead of the id directly:
the FQN of \textit{file1} for the user
\textit{bob} is \textit{file1:bob}. When only the filename is specified, we
implicatively augment it with the uid of the operation.

When we want to lookup an directory entry with a prefix we have three
cases to consider:

\begin{enumerate}
    \item \textbf{There is only one entry for the given prefix}: No conflicts
    concerns this entry, we can simply return it whatever the ViewIDs are.
    \item \textbf{There is a prefix conflict and one entry share the same ViewId}:
    In that case, we assume that the user refers to its own file and resolve it
    as if there was no conflict.
    \item \textbf{There is a prefix conflict and no entry has a matching ViewID}:
    We cannot now for sure what the user referred to, a FQN is required.
\end{enumerate}

This resolution also happen when listing a directory. By default,
when there is no conflict, the user will only see prefix names.

In case of conflict, a prefix is showed for in conflict entries that share
the same ViewId of the user otherwise it is
the FQN. A benefit of this approach is that the user always know when to use
a shorthand or a fully qualified name.

To resolve a conflicts, the user only need to use the standard
FS operations with an FQN or a prefix name. In ElmerFS we didn't need any other
special mechanism.
To give an example,
if a user wants to delete a link named \textit{file1:bob} that is in conflict
with is own entry \textit{file1} a call to \textit{rm file1:bob}
is enough and intuitive.

\section{Experimentation and future explorations}

\subsection{Convergence and overhead.}
\subsection{About conditional transactions ?}

\section{Conclusion}



\end{document}
