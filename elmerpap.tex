%\documentclass[sigconf]{acmart}
\documentclass[sigconf,anonymous,10pt]{acmart}

% Copyright
% Pick the correct copyright notice that matches your rights form
%\setcopyright{none}
\setcopyright{acmcopyright}
%\setcopyright{acmlicensed}
%\setcopyright{rightsretained}
%\setcopyright{usgov}
%\setcopyright{usgovmixed}
%\setcopyright{cagov}
%\setcopyright{cagovmixed}

% DOI
%\acmDOI{10.475/1234.5678}

% ISBN
%\acmISBN{123-4567-24-567/08/06}

%Conference
\acmConference[SYSTOR'21]{ACM International Systems and Storage Conference}{June 14-16, 2021}{Haifa, Israel}
\acmYear{2021}
\copyrightyear{2021}

%\acmArticle{4}
\acmPrice{15.00}

% These commands are optional
%\acmBooktitle{Proccedings of the ACM International Systems and Storage Conference}
%\editor{John Doe}
%\editor{Jane Doe}

\begin{document}

\title{Position: CRDTs for truly concurrent file systems}

% Fill in your details. The \documentclass "anonymous" parameter keeps them hidden, remove it after the review process finishes
\author{Romain Vaillant}
\affiliation{%
  \institution{Scality}
  \city{Paris}
  \state{France}
}
\email{romain.vaillant@scality.com}


% The default list of authors is too long for headers.
\renewcommand{\shortauthors}{R. Vaillant et al.}

\begin{abstract}
This paper addresses the problem of building highly available shared and
geo-replicated distributed POSIX file systems (FS).
We explore the path of building a distributed file system that
is geos-distributed, highly resilient and scalable by
using Conflict-free Replicated Data Types (CRDTs) i.e. replicated data
structures that don’t need any coordination. We build upon these data
structures and explore solutions to ensure that the application remains correct and in a final state that keeps the original user intention in case of conflicting operations.

\end{abstract}

%% Note: Classification and Keywords are only required for the camera-ready version

%
% The code below should be generated by the tool at
% http://dl.acm.org/ccs.cfm
% Please copy and paste the code instead of the example below.
%
\begin{CCSXML}
	<ccs2012>
	   <concept>
		   <concept_id>10010520.10010575.10010755</concept_id>
		   <concept_desc>Computer systems organization~Redundancy</concept_desc>
		   <concept_significance>500</concept_significance>
		   </concept>
	   <concept>
		   <concept_id>10011007.10011074.10011075.10011077</concept_id>
		   <concept_desc>Software and its engineering~Software design engineering</concept_desc>
		   <concept_significance>300</concept_significance>
		   </concept>
	   <concept>
		   <concept_id>10011007.10011074.10011075.10011077</concept_id>
		   <concept_desc>Software and its engineering~Software design engineering</concept_desc>
		   <concept_significance>300</concept_significance>
		   </concept>
	   <concept>
		   <concept_id>10010520.10010521.10010537.10010538</concept_id>
		   <concept_desc>Computer systems organization~Client-server architectures</concept_desc>
		   <concept_significance>300</concept_significance>
		   </concept>
	   <concept>
		   <concept_id>10010520.10010575.10010578</concept_id>
		   <concept_desc>Computer systems organization~Availability</concept_desc>
		   <concept_significance>500</concept_significance>
		   </concept>
	   <concept>
		   <concept_id>10010520.10010575.10010577</concept_id>
		   <concept_desc>Computer systems organization~Reliability</concept_desc>
		   <concept_significance>500</concept_significance>
		   </concept>
	 </ccs2012>
\end{CCSXML}

\ccsdesc[500]{Computer systems organization~Redundancy}
\ccsdesc[300]{Software and its engineering~Software design engineering}
\ccsdesc[300]{Software and its engineering~Software design engineering}
\ccsdesc[300]{Computer systems organization~Client-server architectures}
\ccsdesc[500]{Computer systems organization~Availability}
\ccsdesc[500]{Computer systems organization~Reliability}

\keywords{ACM proceedings, \LaTeX, text tagging}

\maketitle

\sloppy

\section{Introduction}

% Modern services rely on infrastructures composed of multiple geographically
% distributed data centers for  ensuring fault-tolerance and
% achieving high performance and scalability.

System designers have long sought to design geo-replicated file systems
allowing concurrent updates on multiple replicas (active-active replication)
that are correct, remain available under partitions,
and provide high performances (Bell Labs P9~\cite{pike1995plan}, AFS~\cite{howard1988scale}).

One approach is to rely on strong consistency, ensuring that replicas agree on a single, global order of operations. Ordering operations across replicas, however, requires coordination among geographically distributed nodes, and therefore comes at the cost of performance overheads and reduced availability under network partitions.

An alternative approach is to provide weak consistency guarantees.
This approach does not require coordination among replicas, and ensures availability under network partitions.

However, resolving conflicts while ensuring that file system invariants are maintained and user operations not discarded is hard.

Existing systems that accept concurrent updates on multiple replicas without coordination fails both to converge to a predictable state and
to keep the user's original intention when the operation was issued~\cite{cai2018some}.
Convergence of those systems can leads to directory being merged,
files being lost or have their content mixed in unexpected manners.

It is difficult for a user to build a simple intuition of these behaviors,
leading to misconceptions on what their tool offers and how it react in these situations~\cite{tang2013you}

In this paper, we present ElmerFS, a POSIX-compliant geo-replicated
file system, that allow users to resolve conflicts using known FS operations.

ElmerFS does not require coordination between replicas, ensures correctness in the face of concurrent conflicting updates and remains available under network partition and server crashes.

The design of ElmerFS is based on Conflict-free Replicated Data Types (CRDTs), replicated data structures that ensure that modifications performed on different replicas can always be merged into a consistent state.

\section{Designing a file system}

\subsection{File system invariants}

POSIX's file systems organize files and directories in a tree structure.
To design our file system we focus on the following tree invariants:

\begin{enumerate}
\item Tree nodes are always reachable from the root directory.
\item Directories can have at most one parent node.
\item Cycles are possible only through symbolic links.
\end{enumerate}

We categorize two types of operations.
Structural operations, which can be used to manipulate the file system
structure (the tree), and content operations,
which can be used to store and retrieve information from files.

File systems are commonly represented internally as a collection of inodes.
We don't depart from this representation:
An inode represents a file system object (file, directory or symbolic link).
Each inode is identified by a unique numeric identifier (an ino).
A link is a reference to an inode. A directory entry is a named and visible representation of a link.

\subsection{File systems under weak consistency}
\label{fs:weak}

Preserving file system invariants in a replicated file system
that allows updates in multiple replicas with coordination
among them presents several design issues:

\begin{itemize}
	\item \textbf{Unique identifiers}: Any operation that creates
	inodes needs to generate a unique identifier.
	Without coordination among replicas, generated ids might conflict.
	In practice, this is addressed using 16 byte ids.
	However, 8 byte ids are required in POSIX.
	\item \textbf{Named links}: Operations that create or move links
	(rename, link and creation operations) may result in conflicts
	in which concurrent operations οn different replicas create
	links with the same name in the same directory.
	Systems may resolve this type of conflict by automatically
	renaming those links, or expose the conflict resolution to the user.
	\item \textbf{Preserving the FS tree invariants}: Concurrent rename (move)
	operations without coordination may result in cycles or unexpected
	additional links. Merging an operation that moves a directory A into a
	directory B with a concurrent operation that moves B into A violates the
	FS invariants as it results in a cycle.
	Furthermore, merging two concurrent operations that move the same directory
	to different destinations results in a directory with two parents,
	which violates the FS invariants.
	\item \textbf{Deletion of inodes}: Concurrent link and unlink operations may result in different behavior depending on the order in which operations are applied on each replica. The FS may be left with a partial inode that is not fully deleted but apparent.
	\item \textbf{Permissions changes}: Updating permission from a replica may not be propagated directly to other replicas. Ensuring Operations that have no longer sufficient privileges should be discarded.
\end{itemize}

\subsection{What are CRDTs and a brief overview of AntidoteDB}

Ensuring that all replicas converge to the same state without coordination
is not trivial.

Conflict-Free Replicated Data types are data structures that
can be replicated across multiple replicas, and these replicas can be updated independently and concurrently without coordination.

By construction, CRDTs guarantee that modifications on different
replicas can always be merged into a consistent state without requiring
any special conflict resolution code or user intervention.

There are two approaches to CRDTs: Operation-based CRDTs replicate their state
by transmitting only the update operation that are commutative.
State-based CRDTs send their full local state to other replicas and merge
incoming states with a commutative, associative and idempotent function~\cite{shapiro2011conflict}.

For example, one of the simplest operation based CRDT is the counter.
An unbounded counter is a simple local integer whose operations are increment
and decrement.
Addition and subtraction commutes, no matter in which order the operations
arrived, the replicas update their local integer and will converge to the
same value.

AntidoteDB is a key-value store, where each value is a CRDT. AntidoteDB
provides us a library of needed CRDT types.
This means that the designer must map structures to the appropriate CRDT types
and ensure that he can express the application’s operations with the set
of operations offered by the chosen CRDT types.

AntidoteDB implements for us the replication protocol and the
persistence and acts as the only source of truth for our state.
From the library of available CRDTs, we mainly use three kind:

\begin{itemize}
	\item \textbf{Remove Win Set} (RWSet): A set data structure with add and
		  remove operations. When a conflicting concurrent add and remove
		  are issued (of entries that are equal), the add operation is discarded.
	\item \textbf{Remove Win Map} (RWMap): A map data structure that have also
		  favor remove operation when conflict updates arise.
	\item \textbf{Last Writer Win Register} (LWWR): A blob of data which
	keep only the last update issued.
\end{itemize}

\subsection{Assumptions and objectives}

We want to leverage CRDTs to develop a file system that is always available
and that provides good response times whatever the network conditions.
It must support active/active configurations
(i.e. two geographically distant clusters can issue read, write and structural
operation at the same time without coordination with each other).

Given the requirements, we want to keep the behavior of the file system as
close as possible to a local file system.
In summary, we want the following properties:

\begin{itemize}
	\item \textbf{Preserve intention}: We don’t want to perform too many actions
	or changes besides what the ones the user does. We want the user to be able
	to develop a simple mental model to understand the underlying convergence
	properties.
	\item \textbf{Truly concurrent operations}: One way to handle concurrency
	is to serialize operations applied on the relevant objects.
	CRDTs avoid this and allow true concurrency without the need for
	a consensus.
	\item \textbf{Follow the POSIX standard}.
	\item \textbf{Atomic operation}: No matter how complex a FS operation is,
		  it should be either performed are completely discarded.
	\item \textbf{Always available}: CRDT types allow us to design systems that
	are always available even under extreme network conditions.
	\item \textbf{Active-Active}: Several replicas accept operations
	(structural and updates) concurrently and propagate them from one another,
	even after long delays.
\end{itemize}

Focus has been put on building on CRDTs to create a highly resilient and truly
concurrent file system that follows the strict POSIX invariants
while providing users a simple interface to deal with conflicting updates.

\section{System Overview}

\subsection{The layered architecture of ElmerFS}

From a hight level point of view, a ElmerFS deployment consists of a number
of data centers.

Each data center holds a full replica of the file system.
Clients communicate with the data center nearest to them.
All reads and writes are served by the local data center, without the need
for coordination across data centers.
Updates are asynchronously propagated between data centers.
Data centers can continue serving user requests even if connectivity
with other data centers is lost due to network partitions.

Within a data center, an ElmerFS cluster consists of an arbitrary number
of nodes, and uses a shared-nothing architecture.
Unlike other storage systems in the industry, there is no minimal
requirement concerning the number of nodes in a cluster. User requests are
served as long as their is an available node in the data center.

A node in a data center is an ElmerFS process, which consists of the following
layers.

\subsection{Interface}

The interface layer is responsible for handling the user interaction
with the file system.

It is based on the FUSE protocol, a user-space protocol to implements
file systems. It receives FUSE requests,
matches them with the corresponding operation available in
the translation layer, and creates the appropriate response.
This layer implements the standard FS operations.

ElmerFS is multi-threaded and asynchronous. Each FUSE request spawns an
independent task that runs concurrently with other tasks.

\subsection{Translation}

The translation layer is responsible for translating FUSE requests
to CRDT operations. It translates each high-level FS operation
to a collection of operations on CRDTs.

All CRDT operations corresponding to a specific FS operation are bundled into
a transaction. This ensures that FS operations are atomic.

\subsection{Transactional CRDT}

Requests from the translation layers are passed to the transactional CRDT
layer that ensures that CRDTs are persisted.
This layer is also responsible for replicating CRDTs across data centers.
ElmerFS relies on AntidoteDB for implementing this layer.

\subsection{Modeling the file system using CRDTs}

ElmerFS represents the state of the file system using CRDTs.
In particular, it models four main entities: inode objects,
symbolic links, blocks, and directories.
An inode structure stores metadata for an inode in the file system.
We represent the inode structure using a map data type (RWMap).

In ElmerFS, files are sparse (gaps are allowed).
We use a register data type (LWReg) of fixed size file blocks.
We only allocate blocks for ranges that have been written.

We represent symbolic links as a special case of files that store only
the target path.

We represent each directory using a RWSet. Directory entries in the set are
ino - name pairs.

The file system hierarchy is implicit. A parent directory contains its child
directories and a child directory keeps a pointer to its parent through
the special ".." file.

\section{Remaining correct}

CRDTs ensure Strong Eventual Consistency (SEC)~\cite{shapiro2011conflict} but
do not ensure that the application invariants remains correct
nor it ensures that convergence lead to a state that is meaningful
to the user.

The challenge is to keep those invariant always correct under any sequence
of operation while ensuring that no data or user intention is lost through
unintended conflict resolution.

When translating our high level operations, we had to consider problems
stated in section 2 and chose the proper CRDT types,
what metadata needs to be stored and what sequence of
CRDT operations needs to be issued to remain correct.

\subsection{Generating the inode number}

To address this, we use a global counter. Access to this counter
is serialized through a distributed lock.

To reduce the overhead of the contention on this lock,
each time that we access the global counter we reserve a range of ino that
we will then consume locally.
Note that we also don’t recycle ino of deleted inodes,
we can’t assert that all replicas have converged to a state where the ino is
not used anymore. Generating 100 000 files per second, it would take around
8 million years to exhaust this counter.

\subsection{Ensuring deletion}

To ensure that delete operations are honored, we chose to use map and set that
honors concurrent removes.

Favoring concurrent remove operations in our file systems means
that when an unlink is concurrent with a link or a rename,
the unlink will prevail when replicas will converge.

We can only update some key of the map and issue a remove operation on all
possible existing keys (which is a finite amount when we are mapping
structures) to ensure that the map will never converge to a partial state.

\subsection{Keeping the user intention}

In ElmerFS, we allow names conflict to happen and we expect the user solves
those conflicts using standard FS operations that he is familiar with.

To be able to distinguish between two inodes sharing the same name under
the same parent directory, we use an additional internal unique identifier,
the ViewID.

Apart from being unique, there is no particular requirement for this identifier.
We chose to use the file system user id (uid) and we expect that a
user won’t issue operations from two different processes.

Each time a user creates a link, we not only store the name and the ino
of the link but also the ViewID of the user that created it.
Because we use RWSet for directories with an equality put on names,
entries that would have been previously considered the same
(sharing the same name and not necessarily the same ino) are now distinct.

To interface with the user, we use the concept of partial and
Fully Qualified Names (FQNs).
Partial names are how the user named the link,
FQN are partial names concatenated with the ViewID.
Since the ViewID is unique, we know that all visible links are uniquely
identifiable. Meaning that at anytime the user can chose to refer to its file
from the partial name of the FQN.

In a situation where no conflict exists, we always show partial names.
When there is a conflict, we only show FQN of entries that have a ViewID that
doesn’t match the one used by the user.

Showing only FQN when the ViewID mismatch allows the user to continue to
work by name on the file while other conflicting operations might be merged.
We always favor the ViewID of the user that issued the request.

For example, in a situation with two users,
Bob and Alice where both have created a file name "us".
If Bob tries to list the directory, he will see two files: "us" and "us:alice".
To him "us" is still its own file and we do not lose any content when the CRDTs
are merged.

\subsection{Divergent renames}

Without coordination, \textit{mkdir} and \textit{mknod} are not the only
operations that can create a link. (See section~\ref{fs:weak} for problematic
renames).

Because of this situation, using a counter to track the number of links is not
enough anymore because at the time we issue the \textit{rename} operation
we cannot know if the operation will be concurrent.

To keep track of the number of links, we use another RWSet that is
always updated alongside the directory entries updates.

Each link contains the parent inode number and the FQN which contains
the ViewID.
We use the ViewID again to have the exact same semantics as the set storing
the directory entries. Thus the link set is always valid with respect to
links currently visible in the FS.

However, this solution by itself is not enough to ensure that directories have a
unique parent at all times.

This is solved with an additional Last Writer Win register that serves us as
an arbitrator to decide which link is valid.

Each time we load directories  entries, we check the parent register of each
entry that points to a directory. If the entry comes from a parent that doesn’t
match the value of the register, it is lazily removed and never shown to the user.
Because the register is always updated inside the same transaction, the
value of the register always matches the update that add the entry
to the parent.

\subsection{A simple conflict scenario}

Using the design described above we can imagine how ElmerFS behaves in case
of a conflicting scenario.

To illustrate this, let's take again Bob and Alice. Alice and Bob work
on the same project but can't communicate. They don't want to wait and start to create a
folder named \textit{projectA}. Because the project will likely needs a report,
they both start to create a report file \textit{report.doc} to enter their early work.

Once the connection reestablished, Bob see that there is now two folders:
its own folder \textit{projectA} and another one \textit{projectA:Alice}.
He knowns that Alice must have put useful information in her folder and he
decide to merge the two and simply move all file from \textit{projectA:Alice}.

Alice inspect its folder and sees that Bob wasn't really attentive
and merged their folder even though they had two identical
\textit{report.doc} files.
She sees both \textit{report.doc} and \textit{report.doc:Bob},
adds the modification done by Bob in its file.

After an agreement between Alice and Bob, Bob decide to remove its report file.
Now they both see a unique file tree \textit{projectA/report.doc}.

This sequence is very close to what a user might expect when working with
a local file system.

\section{Future work and exploration ideas}

\label{crdt:future}

\subsection{Concurrent ino number generation and recycling}

Our solution to generate an ino number depicted in 4.1 is not truly concurrent.
To solve this issue and to be able to reuse previously used id,
a specialized CRDT might be needed.
One possible path to construct such CRDT would be to take inspiration
from sequence CRDT used for text processing application where each inserted
character is assigned an unique position (LSEQ~\cite{nedelec2013lseq}, TreeDoc~\cite{letia2009crdts}].

\subsection{Cycle with concurrent rename}

Our implicit hierarchy using map CRDTs does not prevent cycles to be created.
Some CRDT tree design exists [Tree CRDT] but relies on multiple correction
layers that perform additional operations to recover from the broken invariants.
We believe that both this issue could be solved by the use of a conditional transaction framework.
The idea would be to merge operations only if they fulfill a user
provided condition.

For cycles, this would be that the resulting tree does not have a cycle.
Transactions that don't respect the condition should be discarded
in a deterministic manner to ensure convergence.

\subsection{Dealing with Orphan CRDTs}

Our deletion strategy solely relies on issuing a delete operation for all
known CRDT of an entity.

For file content, where we store an implicit and unbounded number of CRDT,
concurrent add operations merges can lead to some content lingering around
without an entity to reference them.

Tombstones are sometimes used in CRDT design, but here we need a mechanism
to link and propagate deletion across multiple CRDT.

We aren’t aware of any protocol that allows this.
A possible framework could rely on a unique tombstone and use conditional
transactions described in the previous section to discard concurrent
add operations.

\section{Conclusion}

In the paper, we explored the design of a truly concurrent shared geo-replicated
file system based on CRDTs.

The solutions that we propose for our problems allow ElmerFS to expose
conflicts in a simple interface while preserving the POSIX FS invariants
for some sequence of operations.

We hope that the solutions described in this paper to ensure that
the application remains correct with only set and maps CRDTs can provide ideas
and can help solving similar convergence issues that might arise
in other applications.

In section~\ref{crdt:future} we see that they are still many paths to explore,
conditional transaction, if they are applicable, might be a solution to some
of these issue. Permissions and CRDTs merge semantics are a
challenging topic~\cite{yanakieva2021access}.

We believe that CRDTs for file system are a path toward highly available
and truly concurrent file systems and hope that more work
goes into this direction.

\bibliographystyle{ACM-Reference-Format}
\bibliography{elmerpap}

\end{document}
